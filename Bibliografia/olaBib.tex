\documentclass{article}
\usepackage[MeX]{polski}
\usepackage[utf8]{inputenc}
\author{Aleksandra Ostrowska}
\title{Bibliografia}
\date{2024}
\begin{document}
\maketitle
\noindent
Bardzo ciekawie napisana i zawierająca dużo wartościowych informacji książka o drzewach. \cite{Drzew} \\
\paragraph{}
Książka, która nigdy nie zostanie opublikowana. \cite{Ola} \\
\section{Palenie papierosów wpływa na układ odpornościowy. Nawet lata po rzuceniu}
Od dawna wiadomo, że palenie papierosów jest szkodliwe, ale naukowcy wciąż odkrywają nieznane dotąd mechanizmy pokazujące, jak zgubny wpływ ma palenie na organizm człowieka. W nowych badaniach naukowcy z Instytutu Pasteura w Paryżu ustalili, że palenie ma wpływ na układ odpornościowy, który utrzymuje się długo po spaleniu ostatniego papierosa.
\paragraph{}
Palenie papierosów zwiększa ryzyko chorób układu krążenia i płuc. Ale uszkadza także wiele innych narządów i nierzadko prowadzi do rozwoju nowotworów, najczęściej płuc, ale też innych typów. Wpływa też na zdrowie psychiczne. W badaniach z 2020 roku badacze ustalili, że palacze częściej cierpią na depresję (więcej na ten temat w tekście: Dym papierosowy negatywnie wpływa na zdrowie psychiczne). Z kolei w analizach z grudnia ubiegłego roku naukowcy wykazali, że palenie może prowadzić do kurczenia się mózgu (więcej w tekście: Badania wskazują, że palenie papierosów powoduje kurczenie się mózgu).
\paragraph{}
Rzucenie palenia zmniejsza ryzyko wystąpienia większości tych chorób, ale, jak ustalili francuscy uczeni, pozostawia ślad w układzie odpornościowym i wydaje się go kształtować na lata. Nowe odkrycia uzupełniają ogromną ilość dowodów na szkodliwe skutki palenia papierosów.
\paragraph{}
Rezultaty oraz opis badań ukazał się na łamach pisma „Nature" \cite{palenieJestzłe}
\section{Ekstrakt z konopi zabija komórki czerniaka}
Australijscy naukowcy wykazali w badaniach in vitro, że ekstrakt z konopi może spowalniać tempo wzrostu komórek czerniaka i zwiększać wskaźnik ich śmierci. Jeśli wyniki uda się powtórzyć w badaniach z udziałem ludzi, może to otworzyć zupełnie nową drogę leczenia nowotworów skóry.
\paragraph{}
Australijscy naukowcy z Charles Darwin University (CDU) oraz z Royal Melbourne Institute of Technology (RMIT) wykazali obiecujące działanie przeciwnowotworowe ekstraktu s konopi siewnych (Cannabis sativa). W badaniach in vitro ekstrakt hamował wzrost komórek nowotworowych czerniaka, czyli nowotworu złośliwego melanocytów, określanego potocznie rakiem skóry. Ekstrakt nie tylko spowalniał tempo wzrostu komórek czerniaka, ale także powodował ich szybszą śmierć.
\paragraph{}
Opis i rezultaty badań ukazały się na łamach pisma „Cells” \cite{Konopie}
\section{Partycypacja społeczna a przemiany systemów i kultur politycznych}
Partycypacja społeczna, realizowana w ramach przygotowywania i zarządzania procesami rozwoju miasta, ukształtowała się na zachodzie Europy w kontekście systemowych przemian politycznych i zmian w zakresie planowania przestrzennego, w latach 70-tych ubiegłego stulecia, a więc w okresie zmierzchu epoki industrialnej, narastającego kryzysu demograficznego oraz przechodzenia od form ekstensywnego rozwoju miast do ich rozwoju jakościowego. W przeszłości udział społeczności lokalnych w planowania i realizacji procesów urbanizacyjnych był ograniczony. Istniała ogólna ogólna zgoda na realizację tą drogą takich potrzeby społeczne, jak
rozwój obszarów przemysłowych generujących nowe miejsca pracy w sytuacji postępującego wzrostu gospodarki oraz obszarów nowego mieszkalnictwa, realizowanych w sytuacji wzrostu demograficznego.
\paragraph{}
Partycypacja społeczna w procesie planowania rozwoju miasta jest elementem systemu planowania, realizowanego w ramach określonej, powszechnie akceptowanej kultury społeczno – politycznej i którego celem jest realizacja uporządkowanego rozwoju przestrzennego, realizowana przy współudziale społeczeństwa obywatelskiego, w oparciu o zasadę zrównoważonego, trwałego i społecznie sprawiedliwego rozwoju.
\paragraph{}
Zasady partycypacji społecznej przeniesione do systemów polityki i planowania nie opartych o takie spójne i ogólnie akceptowane założenia, będą miały z reguły charakter formalny, powierzchowny i nieskuteczny. Z tej też racji, niezależnie od umiejętności wprowadzania określonych technik związanych z partycypacją społeczną, jej skuteczność i pożytek zależy od tego, w ramach jakiej kultury politycznej będzie ona realizowana. \cite{Partycypacja}
\section{OECD}
The OECD PISA-based Test for Schools (PBTS) assessment is designed for use by schools and networks of schools around the world to support research, international benchmarking and school improvement efforts. It collects information about 15-year-old students’ applied knowledge and competencies in reading, mathematics and science, as well as their attitudes toward learning and school. PBTS examines how students in participating schools are prepared to meet the challenges of the future. The data collected by the assessment are an extremely valuable source of information for school principals, educators, and the wider school community.
\paragraph{}
The methodology of the PISA-based Test for Schools is complex and demanding. The PISA-based Test for Schools Technical Report describes those procedures and methodologies along with other features that enable the PISA-based Test for Schools to provide high-quality data to schools and local school administrations wanting to go further in understanding how their own individual schools perform compared with the world’s leading school systems. \cite{zamałotychtestów}
\newpage
\bibliographystyle{plain}
\bibliography{data.bib}
\end{document}